\documentclass[
	a4paper,
	%DIV=13,
	smallheadings,
	%headinclude,
	%footinclude,
	%headsepline,
	%parindent,
	german,
	%captions=tableheading,
	%abstracton
	]
	{scrreprt}
\usepackage{blindtext}
\usepackage[T1]{fontenc}
\usepackage[utf8]{inputenc}
\usepackage[ngerman]{babel}
\usepackage[
	left	= 2 cm,
	right	= 2 cm,
	top		= 2 cm,
	bottom	= 4 cm
	]
	{geometry}
%\typearea[current]{calc}
\usepackage{lmodern}			
\usepackage{microtype}			
\usepackage[font=small,labelfont=bf]{caption}
\usepackage{amsmath,amssymb}
\usepackage{chemformula}
\usepackage{chemfig}
\usepackage{siunitx}
\DeclareSIUnit\year{a}
\usepackage{tabularx} 			
\usepackage{mdwlist}
\usepackage{caption}
\usepackage[babel,german=quotes]{csquotes}
\usepackage{xcolor}
\usepackage{hyperref}
\hypersetup{
		colorlinks	= true,
		linkcolor	= black,
		urlcolor	= blue,
		citecolor	= blue,
		pdftitle    = {Titel},
  		pdfsubject  = {CNC},
  		pdfauthor   = {Christopher Labisch},
  		pdfkeywords = {Tags} ,
  		pdfcreator  = {pdflatex},
  		pdfproducer = {LaTeX with hyperref}
	}
\usepackage[style=numeric-comp]{biblatex}
\def\bibfont{\footnotesize}
\bibliography{../Literatur/LiteraturCNC}

\title{CNC Eigenbau}
\subtitle{Open Source Projekt, Version 1.0}
\date{18.05.2017}
\author{Dawid Konczak, \\ Christopher Labisch, \\ Alexander Lohberg}

\begin{document}


\maketitle
\tableofcontents

\chapter{Open Hardware CNC-Maschinen}
Eine kleine Bildergalerie zu industriellen und DIY-CNC-Maschinen stammt von Roman Black \autocite{:01}.
Er kommentiert einzelne Aspekte der Modelle wie Schlupf von Linearwagen, Antriebsarten, austauschbare Werkzeuge oder verwendete NEMA-Schrittmotoren.



\section{HomoFaciens CNC}
Norbert Heinz (auch bekannt als HomoFaciens) hat einige Prototypen gefertigt.
Hier seien die neusten Entwicklungen angesprochen.
Als Designstudie wurde ein Plotter-Modell aus Pappe gebaut \autocite{:Heinz_CNC_3.0}.
Besonders die Steifigkeit und Geschwindigkeit als Schwachpunkte sind anschaulich erklärt.

In der Version 3.1 werden experimentell verschiedene Linearantriebe betrachtet \autocite{:Heinz_CNC_3.1}.
Diese werden mit einem alten Druckerantrieb, Bürstenmotoren oder Schrittmotoren angetrieben.
Als Sensoren zur Postionsbestimmung kommen Gabellichtschranken, eine optische Maus, oder ein Rotationsencoder zum Einsatz.
Die Ansteuerung erfolgt mit einem Arduino Uno.

In der Version 3.2 wird der Aspekt der Konstruktion näher erläutert \autocite{:Heinz_CNC_3.2}.
Besonders die Bearbeitung der Metallprofile wie Bohren, Biegen, Gewindeschneiden werden erläutert.
Die Spindel wird testweise über verschiedene Übersetzungs-Arten bzw. Zahnradgetriebe angetrieben.
Außerdem wird ein Kühlmittelkreislauf zur Verbesserung der Auflösung in der Tiefe (z-Achse) gezeigt. 
Vor der Inbetriebnahme für einen Test wird der Frästisch über Schrauben feinjustiert.
Nachteilig sind die große Anzahl an Verschraubungen und die mühselige Konstruktion an sich.
Der Preis für die Einzelteile wird mit ungefähr 350€ angegeben. 

\section{RepRap CNC}
Es gibt einige CNC-Versionen auf RepRap-Basis \autocite{:02}.

\section{MicroMill CNC}
Eine kleine CNC-Version im Schreibtischformat wurde kürzlich auf Kickstarter beworben \autocite{:kickstarter}.
Bei diesem Modell wird besonders Wert auf die Einfachheit und Flexibilität gelegt.
Ein besonderes Merkmal ist die Auswahl der Kugel- und Linearlager (Fa. Igus) und die Nutzung eines verfügbaren Wekrzeugs (Fa. Proxxon). 
Der Preis beläuft sich auf \$818, wobei nicht weiter aufgeschlüsselt wird. 

\chapter{Projektmanagement}
Die Herausforderung liegt in dem Projektmanagement, da es sich um ein recht umfangreiches Projekt handelt, das mit wenigen Ressourcen auskommen muss.
Viele Teile benötigt werden und eine lange zu erwartende Produktionsphase steht bevor.
Ein wesentliches Ziel ist die Projektphase so kurz wie möglich zu halten bei Gewährleistung einer guten Produktqualität.

Zu Beginn sind drei Personen involviert, die dieses Projekt unterstützen.
Für den 10.6.2017 ist ein erstes Treffen andauernd, Bei dem konzeptionellen Entwürfe erarbeitet und wichtige Rahmenbedingungen geklärt werden sollen.
Das Projekt als Open Source Hardware ausgelegt ist, soll es au Github dokumentiert werden.
Für meine persönliche Planung würde ich \texttt{Microsoft Project} benutzen, ich kann mir vorstellen, das außer mir niemand dieses Programm benutzt.
Daher wäre es gut, wenn es ein Programm oder eine Plattform gäbe, wo die Ressourcen geplant werden können.

Es könnte eine Art Fahrplan gestellt werden in Form eines Gantt-Diagramms mit Projektphasen und der Definition von Meilensteinen.
Zu Beginn steht die Erarbeitung von Konzepten im Vordergrund, wobei doch schon klar werden soll welche Anforderungen an die Maschine gestellt werden und wie man diese Anforderungen einfach erfüllen kann.
Dies betrifft insbesondere die Materialliste, die möglichst kurz ausfallen soll und die Verfügbarkeit von Materialien.
Ein weiterer wichtiger Aspekt ist die Durchführung der Arbeit an entsprechenden Orten mit entsprechendem Werkzeug.

\section{Projektphasen}
\subsection{Definition von Projektphasen und Meilensteinen}
\subsection{Anforderungsprofil}
\subsection{Recherche}
\subsection{Konzeptionierung}
\subsection{Preisvergleich}

\chapter{Komponenten}
\section{Koordinatentisch}
	\subsection{Aluminiumsystemprofile}
	\subsection{Tisch}
	\subsection{Werkstück-Befestigung}
	\subsection{Kühlmittel-Auffangbecken}
\section{Antrieb}
	\subsection{Antriebswelle}
	\subsection{Antriebsmotoren}
		\section{Bürstenmotoren}
		\section{Schrittmotoren}
\section{Werkzeug}
	\subsection{Werkzeugantrieb}
	\subsection{Bohrmaschine}
		\subsection{Bohrfutter}
		\subsection{Bohrer}
\section{Gerüst}
	\subsection{L-Form}
	\subsection{C-Form}
\subsection{Stahlprofile}
\subsection{Schutz}
\section{Kühlkreislauf}
	\subsection{Kühlmittel}
	\subsection{Schläuche}
	\subsection{Pumpe}
	\subsection{Filter}
\section{Elektronik-Hardware}
	\subsection{Netzteil}
	\subsection{Kabel}
	\subsection{Kabelführung}
	\subsection{Display}
	\subsection{PC-Schnittstelle}
	\subsection{Microcontroller}
	\subsection{Treiber}
	\subsection{Relais}
\section{Elektronik-Software}

\chapter{Methoden}

\section{Modellieren}
	\subsection{FreeCAD}
	\subsection{Illustrator}
	\subsection{Excel}
\section{Konstruktion}
Bei der Konstruktion ist zu beachten, dass die Module nicht zu großen Drehmomenten ausgesetzt sind, da sonst Spiel in die Mechanik kommt.
Deshalb ist es wichtig Torsionskräfte und Hebelkräfte zu kompensieren.
Dazu eignen sich z.B. Fachwerke oder Knotenbleche.
	\subsection{Fachwerk}
	\subsection{Knotenbleche}
Ein alter Dachdecker-Spruch besagt: \quote{Viereck vergeht, Dreieck besteht.}
	\subsection{Stecken}
	\subsection{Langloch-Verbinden}
	\subsection{Schrauben}
		\subsection{Schraubverbindung}
	\subsection{Lager}
		\subsection{Linearlager}
		\subsection{Axiallager}
\section{Bohren}
\section{Schweißen}
	\section{MAK}
\section{Schneiden}
\subsection{Gewindeschneiden}


\printbibliography


\end{document} 